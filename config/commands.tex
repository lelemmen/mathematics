% % % Special notations for sets of numbers
\newcommand{\N}{\mathbb{N}}                                                     % Natural numbers
\newcommand{\R}{\mathbb{R}}                                                     % Real numbers
\newcommand{\C}{\mathbb{C}}                                                     % Complex numbers
\newcommand{\F}{\mathbb{F}}                                                     % General field, R or C

% % % Special notations for groups
\DeclareMathOperator{\GLg}{GL}                                                  % General linear group
\DeclareMathOperator{\SLg}{SL}                                                  % Special linear group
\DeclareMathOperator{\Og}{O}                                                    % Orthogonal group
\DeclareMathOperator{\SOg}{SO}                                                  % Special orthogonal group
\DeclareMathOperator{\Ug}{U}                                                    % Unitary group
\DeclareMathOperator{\SUg}{SU}                                                  % Special unitary group
\DeclareMathOperator{\Autg}{Aut}                                                % The automorphism group

% % % Special notations for algebras
\DeclareMathOperator{\glie}{\mathfrak{g}}                                       % General Lie algebra
\DeclareMathOperator{\gllie}{\mathfrak{gl}}                                     % General linear Lie algebra
\DeclareMathOperator{\sllie}{\mathfrak{sl}}                                     % Special linear Lie algebra
\DeclareMathOperator{\olie}{\mathfrak{o}}                                       % Orthogonal Lie algebra
\DeclareMathOperator{\solie}{\mathfrak{so}}                                     % Special orthogonal Lie algebra
\DeclareMathOperator{\ulie}{\mathfrak{u}}                                       % Unitary Lie algebra
\DeclareMathOperator{\sulie}{\mathfrak{su}}                                     % Special unitary Lie algebra


% % % Other math operators
\DeclareMathOperator{\id}{id}                                                   % Identity function
\DeclareMathOperator{\diag}{diag}                                               % Diagonal matrix



% This essentially duplicates \substack, but adding an alignment point. https://tex.stackexchange.com/a/198806
\makeatletter
\newcommand{\subalign}[1]{%
  \vcenter{%
    \Let@ \restore@math@cr \default@tag
    \baselineskip\fontdimen10 \scriptfont\tw@
    \advance\baselineskip\fontdimen12 \scriptfont\tw@
    \lineskip\thr@@\fontdimen8 \scriptfont\thr@@
    \lineskiplimit\lineskip
    \ialign{\hfil$\m@th\scriptstyle##$&$\m@th\scriptstyle{}##$\crcr
      #1\crcr
    }%
  }
}
\makeatother


% Create a custom command for set notation
%
\DeclarePairedDelimiterX{\set}[1]{\{}{\}}{\setargs{#1}}
\NewDocumentCommand{\setargs}{>{\SplitArgument{1}{;}}m}
{\setargsaux#1}
\NewDocumentCommand{\setargsaux}{mm}
{\IfNoValueTF{#2}{#1} {#1\,\delimsize|\,\mathopen{}#2}}%{#1\:;\:#2}


% Create a custom command for floor
\DeclarePairedDelimiter\floor{\lfloor}{\rfloor}

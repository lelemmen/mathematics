\section{Algebraic structures}
    Algebraic structures are a combination of a set, together with one or more operations, satisfying a list of axioms.

    \subsection{The mathematical definition of a group} \label{sec:group_def}
        A group has a mathematical definition. It is a set $G$\footnote{We often use the same symbol to denote the set of group elements and the actual group. I don't think there is anything wrong with this, as the distinction is most often clear from the context.} with elements $G=\set{g_1, g_2, \dots, g_n}$, together with an operation $\cdot$ (which is often called the group multiplication), meeting the following axioms:
        \begin{enumerate}
            \item closedness
            \begin{equation}
                \forall g_1, g_2 \in G: g_1 \cdot g_2 \in G
            \end{equation}

            \item associativity
            \begin{equation}
                \forall g_1, g_2, g_3 \in G: g_1 \cdot (g_2 \cdot g_3) = (g_1 \cdot g_2) \cdot g_3
            \end{equation}

            \item identity element
            \begin{equation} \label{eq:group_id}
                \exists! \thinspace e \in G: \forall g \in G: e \cdot g = g \cdot e = g
            \end{equation}

            \item inverses
            \begin{equation}
                \forall g \in G: \exists! \thinspace g^{-1} \in G: g \cdot g^{-1} = g^{-1} \cdot g = e
            \end{equation}
        \end{enumerate}
        If the axiom
        \begin{enumerate}
            \setcounter{enumi}{4}
            \item commutativity
            \begin{equation}
                \forall g_1, g_2 \in G: g_1 \cdot g_2 = g_2 \cdot g_1
            \end{equation}
        \end{enumerate}
        is also met, the group is called Abelian. \\

    \subsection{Examples of groups}
        Since the definition of a group is so abstract, let us try to examine some examples of groups. \\

        As a first, let us consider a group that is familiar to all of us. Let us take the set $\mathbb{R}_0$: the rational numbers excluding $0$, together with the operation of multiplication. We can check that every group axiom holds (the identity element is $1$, and we know the inverse of every real number), even the commutative one. We can therefore say that $\mathbb{R}_0$ with multiplication is an Abelian group. \\

        In section \ref{sec:group_def}, we gave a general name to the group operation: group multiplication. This doesn't mean that the group operation can't be addition, for example, as `group multiplication' is just a name. A perfectly valid example of an Abelian group is the set of integers $\mathbb{Z}$, together with addition. Again, we can check that all group axioms hold (the identity element is $0$, and we all know the inverse of integers with respect to addition). \\

        As a slightly more complicated example of a group, we will consider the general linear group over $\mathbb{R}$ of degree $n$, denoted by $\GLg(n, \mathbb{R})$. This is the set of all invertible $n \times n$-matrices with real entries, with the operation of matrix multiplication. The identity element is $I_n$: the $n \times n$-identity matrix (a diagonal matrix with $1$ on the diagonal), and since we have specified the set as being the set of invertible matrices, every matrix has an inverse. We should emphasize that $\GLg(n, \mathbb{R})$ is not Abelian, as, in general, matrix multiplication is not commutative. A special case of this group is formed by requiring that the determinant of the invertible $n \times n$-matrices is equal to $1$. We call this set of matrices, together with matrix multiplication, the special linear group $\SLg(n)$. \\

        Many sets of matrices, together with the operation of multiplication form a group. We have for example $\Og(n)$, being the set of $n \times n$ orthogonal ($Q^\text{T} Q = Q Q^\text{T} = I_n$) matrices under matrix multiplication. A special group that is related to $\Og(n)$ is $\SOg(n)$, being the set of orthogonal matrices with determinant equal to $1$, under matrix multiplication. Furthermore, we also have the group $\Ug(n)$, being the set of $n \times n$ unitary matrices ($U U^\dagger = U^\dagger U = I_n$), under the group operation of matrix multiplication. Again, a special variant is $\SUg(n)$, being the set of $n \times n$ unitary matrices with determinant equal to $1$, under matrix multiplication. \\

        As a first more abstract example, let us take a look at the trivial group. It consists of the set $G = \set{e}$ under group multiplication. We have to specify that $e$ is the element for which equation (\ref{eq:group_id}) holds: $e$ is the identity element, and consequently its own inverse. With this in mind, we can check that the trivial group is Abelian. \\

        As another abstract example, let's take the set of elements
        \begin{equation}
            G = \set{E, C_2, \sigma_v, \sigma_v'} \thinspace ,
        \end{equation}
        with the multiplication table given in Table \ref{table:multiplication_table_C2v}.
        \begin{table}[H] \centering
            \begin{tabular}{r|rrrr}
                            & $E$           & $C_2$         & $\sigma_v$    & $\sigma_v'$   \\ \hline

                $E$         & $E$           & $C_2$         & $\sigma_v$    & $\sigma_v'$   \\
                $C_2$       & $C_2$         & $E$           & $\sigma_v'$   & $\sigma_v$    \\
                $\sigma_v$  & $\sigma_v$    & $\sigma_v'$   & $E$           & $C_2$         \\
                $\sigma_v'$ & $\sigma_v'$   & $\sigma_v$    & $C_2$         & $E$
            \end{tabular}
            \caption{An example multiplication table}
            \label{table:multiplication_table_C2v}
        \end{table}
        A multiplication table is read as follows. Take an element from the first column (for example $E$), and take an element of the second column ($C_2$), and find their product as $C_2 \cdot E = C_2$ (note that we read group multiplication conventionally from right to left). This set $G$, together with the multiplication $\cdot$ specified in the multiplication table, forms a group as all four group axioms are fulfilled. As commutativity is also fulfilled\footnote{An easy way to confirm the commutative property, is to verify that the multiplication table is symmetric with respect to its diagonal.}, this group is even Abelian. \\

        We can even introduce bigger sets:
        \begin{equation}
            G = \set{E, C_3, C^2_3, \sigma_v, \sigma_v', \sigma_v''} \thinspace ,
        \end{equation}
        \begin{table}[H] \centering
            \begin{tabular}{r|rrrrrr}
                            & $E$           & $C_3$         & $C^2_3$       & $\sigma_v$    & $\sigma_v'$   & $\sigma_v''$  \\ \hline

                $E$         & $E$           & $C_3$         & $C^2_3$       & $\sigma_v$    & $\sigma_v'$   & $\sigma_v''$  \\
                $C_3$       & $C_3$         & $C^2_3$       & $E$           & $\sigma_v'$   & $\sigma_v''$  & $\sigma_v$    \\
                $C^2_3$     & $C^2_3$       & $E$           & $C_3$         & $\sigma_v''$  & $\sigma_v$    & $\sigma_v'$   \\
                $\sigma_v$  & $\sigma_v$    & $\sigma_v''$  & $\sigma_v'$   & $E$           & $C^2_3$       & $C_3$         \\
                $\sigma_v'$ & $\sigma_v'$   & $\sigma_v$    & $\sigma_v''$  & $C_3$         & $E$           & $C^2_3$       \\
                $\sigma_v''$& $\sigma_v''$  & $\sigma_v'$   & $\sigma_v$    & $C^2_3$       & $C_3$         & $E$
            \end{tabular}
            \caption{Another example of a multiplication table}
            \label{table:multiplication_table_C3v}
        \end{table}
        Given the multiplication table in Table \ref{table:multiplication_table_C3v}, we can verify that the set $G$, together with the group multiplication forms a non-Abelian group. \\

    \subsection{The mathematical definition of a field}
        A field is a set $\F$ together with two binary operations $+$ and $\cdot$, which fulfills
        \begin{enumerate}
            \item $\F$, together with the operation $+$ is an Abelian group
            \item $\F\backslash\set{0_+}$\footnote{The set $\F$ without the identity element of the operation $+$.}, together with the operation $\cdot$ is an Abelian group,
            \item $\cdot$ is distributive with respect to $+$
        \end{enumerate}
        The last property, distributivity of $\cdot$ over $+$ means the following:
        \begin{align}
            \forall a, b, c \in \F: &a \cdot (b + c) = a \cdot b + a \cdot c \\
            & (a + b) \cdot c = a \cdot c + b \cdot c
        \end{align}

        Some important examples of fields include the field of the real numbers without $0$ ($\R_0$) together with multiplication and addition, and the field of the complex numbers without $0$ ($\C_0$) with the operations addition and multiplication. \\

        In some sense, this distributive property just means that scalar multiplication is a bilinear operation. \\

    \subsection{The mathematical definition of a vector space}
        A vector space over a field $F$ is a set of vectors ($\vb{v} \in V$) together with two binary operations: the vector addition, $+$, and the scalar multiplication with an element of the field, $\cdot$, fulfilling the following axioms:
            \begin{enumerate}
                \item $(V,+)$ is an Abelian group

                \item $V$ is closed with respect to scalar multiplication
                \begin{equation}
                    c \cdot \vb{v} \in V
                \end{equation}

                \item $V$ has an identity element for scalar multiplication
                \begin{equation}
                    1 \in F: 1 \cdot \vb{v} = \vb{v}
                \end{equation}

                \item scalar multiplication is compatible with field multiplication
                \begin{equation}
                    a \cdot (b \cdot \vb{v}) = (ab) \cdot \vb{v}
                \end{equation}

                \item scalar multiplication is distributive over vector addition
                \begin{equation}
                    a \cdot (\vb{u} + \vb{v}) = a \cdot \vb{u} + a \cdot \vb{v}
                \end{equation}

                \item scalar multiplication is distributive over field addition
                \begin{equation}
                    (a + b) \cdot \vb{u} = a \cdot \vb{u} + b \cdot \vb{u}
                \end{equation}
            \end{enumerate}

    \subsection{Examples of vector spaces}
        Every field $\F$ is, in a sense, a vector space over itself, in which scalar multiplication is replaced by the field multiplication, and vector addition is replaced by field addition. \\

        $\R^n$, with elements being column matrices of dimension $n$, together with matrix addition and scalar multiplication, forms a vector space over $\vb{R}$. \\

        $\R^{m \times n}$, with elements being the $(m \times n)$-matrices, together with matrix addition and scalar multiplication, forms a vector space over $\vb{R}$. \\

\section{Bilinear operations}
    Let $U, V, W$ be vector spaces over a field $\F$. A bilinear function is a function
    \begin{equation}
        f: U \times V \rightarrow W: (\vb{u}, \vb{v}) \mapsto f(\vb{u}, \vb{w}) = \vb{w} \thinspace ,
    \end{equation}
    such that $f$ is linear in both of its arguments. This means that $\forall \vb{u}_1, \vb{u}_2 \in U; \forall \vb{v}_1, \vb{v}_2 \in V; \forall a, b, c, d \in \F:$
    \begin{equation}
        f(a \vb{u}_1 + b \vb{u}_2, c \vb{v}_1 + d \vb{v}_2) = ac \thinspace f(\vb{u}_1, \vb{v}_1) + ad \thinspace f(\vb{u_1}, \vb{v}_2) + bc \thinspace f(\vb{u}_2, \vb{v}_1) + bd \thinspace f(\vb{u}_2, \vb{v}_2) \thinspace .
    \end{equation}

    An example of a bilinear map is general matrix multiplication. In the most general case, matrix multiplication is a bilinear map between $\R^{m \times n}$ and $\R^{n \times p}$ to $\R^{m \times p}$. \\

    In the case that $U=V$, and $W$ is the field $\F$ itself, we would talk about a bilinear form. \\

    An example of a bilinear form would be an inner product on $V$. \\

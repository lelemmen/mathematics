\section{Derivatives with respect to vectors}
Let $f(\vb{x})$ be a real-valued scalar field (function). We can then take first-order derivative of the scalar function with respect to the components of $\vb{x}$:
\begin{equation}
    \qty(\pdv{f(\vb{x})}{\vb{x}})_i = \pdv{f(\vb{x})}{x_i} \thinspace ,
\end{equation}
which is a vector that we will call the gradient:
\begin{equation}
    \grad{f(\vb{x})} = \pdv{f(\vb{x})}{\vb{x}} \thinspace .
\end{equation}

Some useful formulas are
\begin{align}
    & \pdv{\vb{x}} (\vb{x}^\text{T} \vb{y}) = \vb{y} \\
    & \pdv{\vb{x}} (\vb{x}^\text{T} \vb{x}) = 2 \vb{x} \\
    & \pdv{\vb{x}} (\vb{x}^\text{T} \vb{A} \vb{y}) = \vb{A} \vb{y} \\
    & \pdv{\vb{x}} (\vb{y}^\text{T} \vb{A} \vb{x}) = \vb{A}^\text{T} \vb{y} \\
    & \pdv{\vb{x}} (\vb{x}^\text{T} \vb{A} \vb{x}) = (\vb{A} + \vb{A}^\text{T}) \vb{x} \thinspace .
\end{align}

We can also calculate second-order (and subsequently higher-order) derivatives of the scalar function with respect to the components of $\vb{x}$. This second-order derivative is a symmetric matrix and is called the Hessian:
\begin{equation}
    \vb{H}(\vb{x})_{ij} = \pdv{f(\vb{x})}{x_i}{x_j} \thinspace .
\end{equation}
\\\

The first-order derivative of a vector field is a matrix and is called the Jacobian:
\begin{equation}
    \vb{J}(\vb{x})_{ij} = \pdv{f_i(\vb{x})}{x_j} \thinspace .
\end{equation}

Some useful formulas are
\begin{align}
    \pdv{\vb{x}}{\vb{x}} = \vb{I} \thinspace .
\end{align}

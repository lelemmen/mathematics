\subsection{Algebras}
    Now that we have defined a bilinear operation, we can continue by adding another, more advanced, algebraic structure called an algebra.

    \subsubsection{The mathematical definition of an algebra}
        If we have a vector space $V$ over a field $\F$, we already have two operations available: vector addition and scalar multiplication. The natural way to extend this concept, is to define a map that combines two vectors into another vector. That is exactly how we end up with an algebra. If now add a bilinear operator $\star$ to a vector space $V$ over $\F$, then we will call $V$ an algebra (with $\star$) over $V$. \\

        In some sense we could say that algebras are a generalization of fields in the way that field multiplication is now generalized to the bilinear operation of the algebra. In a sense, we can call the field multiplication a bilinear operation (in which the vector space associated to the bilinear operation is the field over itself). \\

        Let $\set{\vb{e}_i ; i=1,\dots,n}$ be a basis for the underlying $n$-dimensional vector space $V$ of the algebra. It is then possible, in much the same way as operators can be represented as matrices in a certain basis, to characterize the the multiplication $\star$ of the algebra as
        \begin{equation}
            \vb{e}_i \star \vb{e}_j = \sum_k^n f_{ijk} \vb{e}_k \thinspace ,
        \end{equation}
        in which $f_{ijk}$ are called the structure constants of the algebra. \\

    \subsubsection{Examples of algebras}
        We all know examples of algebras, with the easiest example being the $(n \times n)$-matrices with matrix multiplication. \\

    \subsubsection{The mathematical definition of a Lie algebra}
        A Lie algebra is an algebra $\glie$ over the field $\F$, in which the bilinear operation is the Lie bracket. The Lie bracket $\comm{\cdot}{\cdot}$ is a bilinear function that further obeys
        \begin{enumerate}
            \item alternativity
            \begin{equation}
                \forall x \in \glie: \comm{x}{x} = 0
            \end{equation}

            \item the Jacobi identity
            \begin{equation}
                \forall x, y, z \in \glie: \comm{x}{\comm{y}{z}} + \comm{z}{\comm{x}{y}} + \comm{y}{\comm{z}{x}} = 0
            \end{equation}
        \end{enumerate}

        It can be shown that bilinearity and alternativity together imply anticommutativity:
        \begin{equation}
            \forall x, y \in \glie: \comm{y}{x} = - \comm{x}{y} \thinspace .
        \end{equation}

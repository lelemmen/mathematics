\section{Algebras}
    Now that we have defined a bilinear operation, we can continue by adding another, more advanced, algebraic structure called an algebra.

    \subsection{The mathematical definition of an algebra}
        If we have a vector space $V$ over a field $\F$, we already have two operations available: vector addition and scalar multiplication. The natural way to extend this concept, is to define a map that combines two vectors into another vector. That is exactly how we end up with an algebra. If now add a bilinear operator $\star$ to a vector space $V$ over $\F$, then we will call $V$ an algebra (with $\star$) over $V$. Again, there is an unfortunate notation in which both $V$ represents the algebra, as well as the \\

        In some sense we could say that algebras are a generalization of fields in the way that field multiplication is now generalized to the bilinear operation of the algebra. In a sense, we can call the field multiplication a bilinear operation (in which the vector space associated to the bilinear operation is the field over itself). \\

        Let $\set{\vb{e}_i ; i=1,\dots,n}$ be a basis for the underlying $n$-dimensional vector space $V$ of the algebra. It is then possible, in much the same way as operators can be represented as matrices in a certain basis, to characterize the the multiplication $\star$ of the algebra as
        \begin{equation}
            \vb{e}_i \star \vb{e}_j = \sum_k^n f_{ijk} \vb{e}_k \thinspace ,
        \end{equation}
        in which $f_{ijk}$ are called the structure constants of the algebra. \\

        If $\star$ is associative, i.e.
        \begin{equation}
            \forall \vb{u}, \vb{v}, \vb{w}: \vb{u} \star (\vb{v} \star \vb{w}) = (\vb{u} \star \vb{v}) \star \vb{w} \thinspace ,
        \end{equation}
        then the algebra is called associative.

    \subsection{Examples of algebras}
        We all know examples of algebras, with the easiest example being the $(n \times n)$-matrices with matrix multiplication. \\

    \subsection{The mathematical definition of a Lie algebra}
        A Lie algebra is an algebra $\glie$ over the field $\F$, in which the bilinear operation is the Lie bracket. The Lie bracket $\comm{\cdot}{\cdot}$ is a bilinear function that further obeys
        \begin{enumerate}
            \item alternativity
            \begin{equation}
                \forall T_a \in \glie: \comm{T_a}{T_a} = 0
            \end{equation}

            \item the Jacobi identity
            \begin{equation}
                \forall T_a, T_b, T_c \in \glie: \comm{T_a}{\comm{T_b}{T_c}} + \comm{T_c}{\comm{T_a}{T_b}} + \comm{T_b}{\comm{T_c}{T_a}} = 0
            \end{equation}
        \end{enumerate}

        It can be shown that bilinearity and alternativity together imply anticommutativity:
        \begin{equation}
            \forall T_a, T_b \in \glie: \comm{T_b}{T_a} = - \comm{T_a}{T_b} \thinspace .
        \end{equation}

        In the physics community, the elements $T_a, T_b, \cdots$ are called the generators of the algebra if they are a basis for the underlying vector field. \\

        For the structure constants, the Jacobi identity implies
        \begin{equation}
            \sum_d^n (f_{bcd} f_{ade} + f_{abd} f_{cde} + f_{cad} f_{bde}) = 0 \thinspace .
        \end{equation}

        It is interesting to note that every associative algebra $A$ over a field $\F$ admits a Lie algebra $L(A)$ over the same field $\F$ (both having the same underlying vector space $V$), by defining the Lie bracket as the commutator:
        \begin{equation}
            \comm{T_a}{T_b} = T_a T_b - T_b T_a \thinspace .
        \end{equation}
        The associative algebra $A$ is then called the enveloping algebra of the Lie algebra $L(A)$.

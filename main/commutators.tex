\section{Formulas for commutators and anticommutators}
    When an addition and a multiplication are both defined for all elements of a set $\set{A, B, \dots}$, we can check if multiplication is commutative by calculation the commutator:
    \begin{equation}
        \comm{A}{B} = AB - BA \thinspace .
    \end{equation}
    $A$ and $B$ are said to commute if their commutator is zero. We can analogously define the anticommutator between $A$ and $B$ as
    \begin{equation}
        \comm{A}{B}_+ = AB + BA \thinspace .
    \end{equation}

    From these definitions, we can easily see that
    \begin{align}
        & \comm{A}{B} = - \comm{B}{A} \\
        & \comm{A}{B}_+ = \comm{B}{A}_+ \thinspace .
    \end{align}

    Letting $\dagger$ stand for the Hermitian adjoint, we can write for operators or $A$ and $B$:
    \begin{align}
        & \comm{A}{B}^\dagger = \comm{B^\dagger}{A^\dagger} = - \comm{A^\dagger}{B^\dagger} \\
        & \comm{A}{B}^\dagger_+ = \comm{A^\dagger}{B^\dagger}_+
    \end{align}

    If $U$ is a unitary operator or matrix, we can see that
    \begin{equation}
        \comm{U^\dagger A U}{U^\dagger B U } = U^\dagger \comm{A}{B} U \thinspace .
    \end{equation}

    Using the definitions, we can derive some useful formulas for converting commutators of products to sums of commutators:
    \begin{align}
        & \comm{A}{BC} = B \comm{A}{C} + \comm{A}{B} C \\
        & \comm{AB}{C} = A \comm{B}{C} + \comm{A}{C}B \\
        & \comm{AB}{CD} = A \comm{B}{C} D + AC \comm{B}{D} + \comm{A}{C} DB + C \comm{A}{D} B \\
        & \comm{ABC}{D} = AB \comm{C}{D} + A \comm{B}{D} C + \comm{A}{D} BC \\
        & \comm{A}{BCD} = BC \comm{A}{D} + B \comm{A}{C} D + \comm{A}{B} CD
    \end{align}

    In general, we can summarize these formulas as
    \begin{equation}
        \comm{A}{B_1 B_2 \cdots B_n} = \comm{A}{\prod_{k=1}^n B_k} = \sum_{k=1}^n B_1 \cdots B_{k-1} \comm{A}{B_k} B_{k+1} \cdots B_n \thinspace .
    \end{equation}

    Concerning sufficiently well-behaved functions $f$ of $B$, we can prove that
    \begin{equation}
        \comm{\comm{A}{B}}{B} = 0 \qquad\Rightarrow\qquad \comm{A}{f(B)} = f'(B) \comm{A}{B} \thinspace .
    \end{equation}

    In electronic structure theory, we often want to end up with anticommutators:
    \begin{align}
        & \comm{A}{BC} = \comm{A}{B}_+ C - B \comm{A}{C}_+ \\
        & \comm{AB}{C} = A \comm{B}{C}_+ - \comm{A}{C}_+ B
    \end{align}

    In electronic structure theory, we often end up with anticommutators. In case there are still products inside, we can use the following formulas:
    \begin{align}
        & \comm{A}{BC}_+ = \comm{A}{B} C + B \comm{A}{C}_+ \\
        & \comm{A}{BC}_+ = \comm{A}{B}_+ C - B \comm{A}{C} \\
        & \comm{AB}{C}_+ = A \comm{B}{C}_+ - \comm{A}{C} B \\
        & \comm{AB}{C}_+ = \comm{A}{C}_+ B + A \comm{B}{C}
    \end{align}

    The elementary BCH (Baker-Campbell-Hausdorff) formula reads
    \begin{equation}
        \exp(A) \exp(B) = \exp(A + B + \frac{1}{2} \comm{A}{B} + \cdots) \thinspace ,
    \end{equation}
    where higher order nested commutators have been left out. From this, two special consequences can be formulated:
    \begin{equation}
        \exp(-A) \thinspace B \thinspace \exp(A) = B + \comm{B}{A} + \frac{1}{2!} \comm{\comm{B}{A}}{A} + \cdots \thinspace ,
    \end{equation}
    and
    \begin{equation}
        \exp(A) \thinspace B \thinspace \exp(-A) = B + \comm{A}{B} + \frac{1}{2!} \comm{A}{\comm{A}{B}} + \cdots \thinspace .
    \end{equation}
